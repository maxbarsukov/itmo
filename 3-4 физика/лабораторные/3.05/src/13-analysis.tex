\section{Вывод и анализ результатов}
В результате проведения лабораторной работы были получены графики зависимости электрического сопротивления от температуры для металлического и полупроводникового образцов. Анализ полученных графиков показал, что они имеют хорошее линейное приближение, что позволило вычислить температурный коэффициент сопротивления металла и ширину запрещенной зоны полупроводника.

Исходя из полученного значения ширины запрещенной зоны полупроводника, равного $E_g = 0,66 \text{ эВ}$ с учетом погрешности измерений, можно предположить, что исследуемый полупроводниковый образец является германием. Данное предположение основывается на том, что ширина запрещенной зоны германия при комнатной температуре составляет 0,67 эВ. Таким образом, значение попадает в доверительный интервал.

Температурный коэффициент сопротивления металлического образца был вычислен и составил $\alpha = 4,9 \times 10^{-3} \, \frac{1}{^\circ C}$. Однако, ввиду того, что разброс значений температурного коэффициента сопротивления для различных металлов достаточно велик, однозначно определить материал металлического образца не представляется возможным. Так, например, температурный коэффициент сопротивления вольфрама составляет $4,5 \times 10^{-3} \, \frac{1}{^\circ C}$, что совпадает с полученным в ходе работы значением. В то же время, температурный коэффициент сопротивления других металлов, таких как медь, алюминий или серебро, также находятся в близком диапазоне.

Таким образом, в ходе лабораторной работы были получены графики зависимости электрического сопротивления от температуры для металлического и полупроводникового образцов, по которым были вычислены температурный коэффициент сопротивления металла и ширина запрещенной зоны полупроводника.
%\section{Задания к отчету}
