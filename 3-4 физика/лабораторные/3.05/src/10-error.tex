\section{Расчет погрешностей измерений}

\subsection{Расчет погрешностей для ширины запрещенной зоны}

Ширина запрещенной зоны \( E_{gij} \) рассчитывается по формуле:
\[ 
E_{gij} = 2k \frac{T_i \cdot T_j}{T_j - T_i} \ln\left(\frac{R_i}{R_j}\right)
\]

Среднее значение ширины запрещенной зоны:
\[ 
\langle E_g \rangle = \frac{1}{n}\sum_{i=1}^n E_{gij}
\]

Для погрешности ширины запрещенной зоны используем формулу для стандартного отклонения среднего значения:
\[ 
\Delta E_g = t_{\alpha, n} \sqrt{\frac{\sum_{i=1}^n (E_{gij} - \langle E_g \rangle)^2}{n(n-1)}}
\]
где \( t_{\alpha, n} \) - коэффициент Стьюдента для доверительной вероятности \(\alpha = 0.95\) и числа измерений \( n \).

\subsection{Расчет погрешностей для температурного коэффициента сопротивления}

Температурный коэффициент сопротивления \( \alpha_{ij} \) рассчитывается по формуле:
\[ 
\alpha_{ij} = \frac{R_i - R_j}{R_j \cdot t_i - R_i \cdot t_j}
\]

Среднее значение температурного коэффициента сопротивления:
\[ 
\langle \alpha \rangle = \frac{1}{n} \sum_{i=1}^n \alpha_{ij}
\]

Для погрешности температурного коэффициента сопротивления используем формулу для стандартного отклонения среднего значения:
\[ 
\Delta \alpha = t_{\alpha, n} \sqrt{\frac{\sum_{i=1}^n (\alpha_{ij} - \langle \alpha \rangle)^2}{n(n-1)}}
\]

Погрешность \( \Delta E_g \):
\[ 
\Delta E_g = 2.45 \sqrt{\frac{(0.90 - 1.05)^2 + (0.96 - 1.05)^2 + (0.97 - 1.05)^2 + (1.40 - 1.05)^2 }{4(4-1)}} \times 10^{-19} \, \text{Дж}
\]
\[= 0.15 \times 10^{-19} \, \text{Дж}
\]

Погрешность \( \Delta \alpha \):
\[ 
\Delta \alpha = 2.45 \sqrt{\frac{(5.64 - 4.9)^2 + (5.11 - 4.9)^2 + (4.69 - 4.9)^2 + (4.56 - 4.9)^2 + (4.48 - 4.9)^2}{5(5-1)}} \times 10^{-3} \, \frac{1}{^\circ C} 
\]
\[
=0.11 \times 10^{-3} \, \frac{1}{^\circ C} 
\]