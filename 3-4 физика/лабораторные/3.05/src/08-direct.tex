\clearpage
\section{Результаты прямых измерений и их обработки}
\begin{table}[!ht]
    \centering
    \begin{tabular}{|l|l|l|l|l|l|c|}
    \hline
        \text{№} & \textit{T, K} & \textit{I, мкА} & \textit{U, В} & \textit{R, Ом} & ln(R) & \textit{$T^{-1}$, 10³/K } \\ \hline
        %1 & 270 & 1234 & 0,924 & 748,78   & 6,61 & 3,70  \\ \hline   % >>> горляков 
        %2 & 275 & 1337 & 0,84 &  628,27   & 6,44 & 3,63   \\ \hline  % >>> ходжаев
        3 & 280 & 1460 & 0,743 & 508,90   & 6,23 & 3,57  \\ \hline   % >>> барсуков
        4 & 285 & 1573 & 0,659 & 418,94   & 6,03 & 3,50   \\ \hline
        5 & 290 & 1676 & 0,578 & 344,86   & 5,84 & 3,44  \\ \hline
        6 & 295 & 1765 & 0,509 & 288,38   & 5,66 & 3,38  \\ \hline
        7 & 300 & 1850 & 0,433 & 234,05   & 5,45 & 3,33  \\ \hline
        8 & 305 & 1926 & 0,362 & 187,95   & 5,23 & 3,27  \\ \hline   % <<< горляков
        9 & 310 & 2210 & 0,349 & 157,91 & 5,06 & 3,22 \\ \hline % <<< ходжаев 
        10 & 315 & 2350 & 0,228 & 97,02 & 4,57 & 3,17 \\ \hline  % <<< барсуков
    \end{tabular}
    \caption{Результаты измерений и их обработки для полупроводникового образца}
\end{table}
\begin{table}[!ht]
    \centering
    \begin{tabular}{|l|l|l|l|l|l|}
    \hline
        \text{№} & \textit{T, K} & \textit{I, мкА} & \textit{U, В} & \textit{R, кОм} & \textit{t, °C } \\ \hline
        %1 & 350 & 1006 & 1,571 & 1561,63 & 76,85 \\ \hline % >>> горляков
        %2 & 345 & 1015 & 1,545 & 1522,16 & 71,85 \\ \hline % >>> ходжаев
        3 & 340 & 1022 & 1,538 & 1504,89 & 66,85 \\ \hline % >>> барсуков
        4 & 335 & 1038 & 1,525 & 1469,17 & 61,85 \\ \hline
        5 & 330 & 1053 & 1,513 & 1436,84 & 56,85 \\ \hline
        6 & 325 & 1067 & 1,502 & 1407,68 & 51,85 \\ \hline
        7 & 320 & 1080 & 1,492 & 1381,48 & 46,85 \\ \hline
        8 & 315 & 1093 & 1,481 & 1354,98 & 41,85 \\ \hline
        9 & 310 & 1105 & 1,470  & 1330,31 & 36,85 \\ \hline  % <<< горляков
        10 & 305 & 1119 & 1,459 & 1303,842717 & 31,85 \\ \hline% <<< ходжаев
        11 & 300 & 1132 & 1,448 & 1279,151943 & 26,85 \\ \hline% <<< барсуков
    \end{tabular}
    \caption{Результаты измерений и их обработки для металлического образца}
\end{table}

% Полупроводниковый образец
\subsection*{Полупроводниковый образец}

\subsubsection*{Расчёт сопротивления $R$ для каждого измерения:}
\[
R = \frac{U}{I} = \frac{0.924 \, \text{V}}{1234 \times 10^{-6} \, \text{A}} = 748.78 \, \Omega
\]

\subsubsection*{Расчёт натурального логарифма сопротивления $\ln(R)$ для каждого измерения:}
\[
\ln(R) = \ln(748.78) \approx 6.61
\]

\subsubsection*{Расчёт величины обратного значения температуры $\frac{10^3}{T}$:}
\[
\frac{10^3}{T} = \frac{10^3}{270} \approx 3.70 \, \frac{1}{\text{K}}
\]

% Металлический образец
\subsection*{Металлический образец}

\subsubsection*{Расчёт сопротивления $R$ для каждого измерения:}
\[
R = \frac{U}{I} = \frac{1.571 \, \text{V}}{1006 \times 10^{-6} \, \text{A}} = 1561.63 \, \Omega
\]

\subsubsection*{Расчёт температуры $t$ по шкале Цельсия:}
\[
t = T - 273 = 350 - 273,15 = 66,85 \, ^\circ\text{C}
\]